\documentclass[12pt,a4paper]{article}
\usepackage[latin1]{inputenc}
\usepackage{float}
\usepackage{amsmath}
\usepackage{amsfonts}
\usepackage{amssymb}
\usepackage{graphicx}
\usepackage[hidelinks]{hyperref}

\author{Giancarlo Danese - 945265}
\author{Davide Savoldelli - } 
\date{A.Y. 2019/2020 - Prof. Di Nitto Elisabetta}


\title{
 \textbf{\Huge{SafeStreets}} \\
 \large Requirements Analysis and Specification Document
}

\begin {document}

 \begin{figure}
  \centering
  \includegraphics[width=1.0\linewidth]{C:/Users/Utente/Documents/logo_poli.jpg}
 \end{figure}

 \maketitle
 \newpage
 \tableofcontents
 \newpage

\section{INTRODUCTION}
\subsection{Purpose} 
This documents represents the Requirements Analysis and Specification Document (RASD) for the SafeStreets software project.
The goal of this project is to deploy a system that allows users to report traffic/street violations to authorities. Users will have the possibility to send pictures of the violation attached with the date/time and position of the vehicle committing the violation. The software focuses particularly on parking violations, including double parking or parking in a disabled-reserved spot.
\subsection{Scope}
SafeStreets will have an embedded algorithm which will analyze pictures of the vehicle plates sent by the user in order to recognize the vehicle. This information, together with the position of the vehicle and the type of violation that has been committed, will be stored in the software's database. Both end users and mostly authorities will have the chance to mine the information retrieved in the database by highlighting the streets/areas in which most of the violations are committed, the type of vehicles which commit most of the violations and which type of violations occur the most. 
\subsection{Definitions, Acronyms, Abbreviations} 
\subsection{Revision history} 
\subsection{Reference Documents} 
\subsection{Document Structure} 
\section{OVERALL DESCRIPTION}
\subsection{Product perspective: here we include  further details on  the  shared phenomena and a 
domain model (class diagrams and statecharts)} 
\subsection{Product functions: here we include the most important requirements} 
\subsection{User characteristics: here we include anything that is relevant to clarify their needs} 
\subsection{Assumptions, dependencies and constraints: here we include domain assumptions } 
\section{SPECIFIC REQUIREMENTS: Here we include more details on all aspects in Section 2 if they 
can be useful for the development team}
\subsection{External Interface Requirements} 
\subsubsection{User Interfaces}
\subsubsection{Hardware Interfaces}
\subsubsection{Software Interfaces}
\subsubsection{Communication Interfaces}
\subsection{Functional  Requirements:  Definition  of  use  case  diagrams,  use  cases  and  associated 
sequence/activity diagrams, and mapping on requirements} 
\subsection{Performance Requirements} 
\subsection{Design Constraints}
\subsubsection{Standards compliance}
\subsubsection{Hardware limitations}
\subsubsection{Any other constraint} 
\subsection{Software System Attributes} 
\subsubsection{Reliability}
\subsubsection{Availability}
\subsubsection{Security}
\subsubsection{Maintainability}
\subsubsection{Portability}
\section{FORMAL ANALYSIS USING ALLOY: This section should include a brief presentation of the 
main objectives driving the formal modeling activity, as well as a description of the model 
itself, what can be proved with it, and why what is proved is important given the problem at 
hand. To show  the soundness and correctness of the model,  this section can show some
worlds obtained by running it, and/or the results of the checks performed on meaningful 
assertions}
\section{EFFORT SPENT: In this section you will include information about the number of hours each 
group member has worked for this document.}
\section{REFERENCES}


\end {document}